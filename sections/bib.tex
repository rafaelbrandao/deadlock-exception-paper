\chapter{Bibliography}

\begin{thebibliography}{4}

\bibitem{lu} Lu, Shan, et al: Learning from mistakes: a comprehensive study on real world concurrency bug characteristics.
ACM Sigplan Notices. Vol. 43. No. 3. ACM, 2008.

\bibitem{singhal} Singhal, Mukesh. Deadlock detection in distributed systems.
Computer 22.11 (1989): 37-48.

\bibitem{knapp} Knapp, Edgar: Deadlock detection in distributed databases.
ACM Computing Surveys (CSUR) 19.4 (1987): 303-328.

\bibitem{marino} Marino, Daniel, et al: Detecting deadlock in programs with data-centric synchronization.
Software Engineering (ICSE), 2013 35th International Conference on. IEEE, 2013.

\bibitem{dolby} Dolby, Julian, et al: A data-centric approach to synchronization.
ACM Transactions on Programming Languages and Systems (TOPLAS) 34.1 (2012): 4.

\bibitem{dawson} Dawson Engler and Ken Ashcraft. RacerX: effective, static detection of race conditions and
deadlocks. SIGOPS Operating Systems Review, 37(5):237–252, 2003.

\bibitem{chand} Chandrasekhar Boyapati, Robert Lee, and Martin Rinard. Ownership types for safe programming:
preventing data races and deadlocks. In OOPSLA ’02: Proceedings of the 17th
ACM SIGPLAN conference on Object-Oriented Programming, Systems, Languages, and Applications,
pages 211–230. ACM, 2002.

\bibitem{praun} C. Von Praun. Detecting synchronization defects in multi-threaded object-oriented programs.
In PhD Thesis, 2004.

\bibitem{vivek} Vivek K. Shanbhag. Deadlock-detection in java-library using static-analysis. Asia-Pacific
Software Engineering Conference, 0:361–368, 2008.

\bibitem{cormac} Cormac Flanagan, K. Rustan M. Leino, Mark Lillibridge, Greg Nelson, James B. Saxe, and
Raymie Stata. Extended static checking for java. In PLDI ’02: Proceedings of the ACM
SIGPLAN 2002 Conference on Programming Language Design and Implementation, pages
234–245. ACM, 2002.

\bibitem{williams} Amy Williams, William Thies, and Michael D. Ernst. Static deadlock detection for java
libraries. In ECOOP 2005 - Object-Oriented Programming, pages 602–629, 2005.

\bibitem{mcsdk} Da Luo, Zhi, Raja Das, and Yao Qi: Multicore sdk: a practical and efficient deadlock detector for real-world applications.
Software Testing, Verification and Validation (ICST), 2011 IEEE Fourth International Conference on. IEEE, 2011.

\bibitem{tanter} Tanter, Éric, et al: Altering Java semantics via bytecode manipulation.
Generative Programming and Component Engineering. Springer Berlin Heidelberg, 2002.

\bibitem{contest} ConcurrentTesting - Advanced Testing for Multi-Threaded Java Applications, \url{https://www.research.ibm.com/haifa/projects/verification/contest/}

\bibitem{magicfuzzer} Cai, Yan, and W. K. Chan: MagicFuzzer: scalable deadlock detection for large-scale applications.
Proceedings of the 2012 International Conference on Software Engineering. IEEE Press, 2012.

\bibitem{pin} Luk, Chi-Keung, et al. Pin: building customized program analysis tools with dynamic instrumentation.
ACM Sigplan Notices 40.6 (2005): 190-200.

\bibitem{mysql} MySQL, available at: \url{http://www.mysql.com}

\bibitem{firefox} Firefox, available at \url{http://www.mozilla.org/firefox}

\bibitem{chromium} Chromium, available at \url{http://code.google.com/chromium}

\bibitem{sammati} Pyla, Hari K., and Srinidhi Varadarajan: Avoiding deadlock avoidance.
Proceedings of the 19th international conference on Parallel architectures and compilation techniques. ACM, 2010.

\bibitem{serenity} Hari K. Pyla and Srinidhi Varadarajan. Transparent Runtime Deadlock Elimination. In
Proceedings of the 21st international conference on Parallel architectures and compilation
techniques, PACT ’12, pages 477–478, New York, NY, USA, 2012. ACM.

\bibitem{pyla} Pyla, Hari Krishna. "Safe Concurrent Programming and Execution." (2013).

\bibitem{rx} F. Qin, J. Tucek, Y. Zhou, and J. Sundaresan. Rx: Treating bugs as
allergiesa safe method to survive software failures. ACM Trans. Comput.
Syst., 25(3), Aug. 2007.

\bibitem{berger} Berger, Emery D., et al: Grace: Safe multithreaded programming for C/C++.
ACM Sigplan Notices. Vol. 44. No. 10. ACM, 2009.

\bibitem{splash} SPLASH-2. SPLASH-2 benchmark suite, available at \url{http://www.capsl.udel.edu/splash}

\bibitem{phoenix} Ranger, Colby, et al: Evaluating mapreduce for multi-core and multiprocessor systems.
High Performance Computer Architecture, 2007. HPCA 2007. IEEE 13th International Symposium on. IEEE, 2007.

\bibitem{grechanik} Grechanik, Mark, et al: Preventing database deadlocks in applications.
Proceedings of the 2013 9th Joint Meeting on Foundations of Software Engineering. ACM, 2013.

\bibitem{ian} Ian Pye. 2011. Locks, deadlocks and abstractions: experiences with multi-threaded programming at CloudFlare, Inc.. In Proceedings of the compilation of the co-located workshops on DSM'11, TMC'11, AGERE!'11, AOOPES'11, NEAT'11, & VMIL'11 (SPLASH '11 Workshops). ACM, New York, NY, USA, 129-132.

\bibitem{hansen} Hansen, Per Brinch. "The programming language concurrent Pascal." Software Engineering, IEEE Transactions on 2 (1975): 199-207.

\bibitem{sanchez} Iván Sanchez. Latin Squares and Its Applications on Software Engineering. Master's thesis, Federal University of Pernambuco, Recife, Brazil, 2011.

\bibitem{paola} Paola Accioly. Comparing Different Testing Strategies for Software Product Lines. Master's thesis, Federal University of Pernambuco, Recife, Brazil, 2012.

\bibitem{buse} Raymond P. L. Buse, et al. Benefits and barriers of user evaluation in software engineering research. ACM SIGPLAN Notices, 46(10):643-656, October 2011.

\bibitem{runes} Per Runeson. Using students as experiement subjects - an analysis on graduate and freshmen student data. In Proceedings of the 7th International Conference on Empirical Assessment in Software Engineering. Keele University, UK, pages 95-102, 2003.

\bibitem{staron} Miroslaw Staron. Using students as subjects in experiments - A quantitative analysis of the influence of experimentation on students' learning process. In CSEE&T, apges 221-228. IEEE Computer Society, 2007.

\bibitem{box} G. E. P. Box, J. S. Hunter, and W. G. Hunter, Statistics for experimenters: design, innovation, and discovery. Wiley-Interscience, 2005.

\end{thebibliography}