\chapter{Bibliography}

\begin{thebibliography}{4}

\bibitem{lu} Lu, Shan, et al: Learning from mistakes: a comprehensive study on real world concurrency bug characteristics.
ACM Sigplan Notices. Vol. 43. No. 3. ACM, 2008.

\bibitem{singhal} Singhal, Mukesh. Deadlock detection in distributed systems.
Computer 22.11 (1989): 37-48.

\bibitem{knapp} Knapp, Edgar: Deadlock detection in distributed databases.
ACM Computing Surveys (CSUR) 19.4 (1987): 303-328.

\bibitem{marino} Marino, Daniel, et al: Detecting deadlock in programs with data-centric synchronization.
Software Engineering (ICSE), 2013 35th International Conference on. IEEE, 2013.

\bibitem{dolby} Dolby, Julian, et al: A data-centric approach to synchronization.
ACM Transactions on Programming Languages and Systems (TOPLAS) 34.1 (2012): 4.

\bibitem{mcsdk} Da Luo, Zhi, Raja Das, and Yao Qi: Multicore sdk: a practical and efficient deadlock detector for real-world applications.
Software Testing, Verification and Validation (ICST), 2011 IEEE Fourth International Conference on. IEEE, 2011.

\bibitem{tanter} Tanter, Éric, et al: Altering Java semantics via bytecode manipulation.
Generative Programming and Component Engineering. Springer Berlin Heidelberg, 2002.

\bibitem{contest} ConcurrentTesting - Advanced Testing for Multi-Threaded Java Applications, \url{https://www.research.ibm.com/haifa/projects/verification/contest/}

\bibitem{magicfuzzer} Cai, Yan, and W. K. Chan: MagicFuzzer: scalable deadlock detection for large-scale applications.
Proceedings of the 2012 International Conference on Software Engineering. IEEE Press, 2012.

\bibitem{pin} Luk, Chi-Keung, et al. Pin: building customized program analysis tools with dynamic instrumentation.
ACM Sigplan Notices 40.6 (2005): 190-200.

\bibitem{mysql} MySQL, available at: \url{http://www.mysql.com}

\bibitem{firefox} Firefox, available at \url{http://www.mozilla.org/firefox}

\bibitem{chromium} Chromium, available at \url{http://code.google.com/chromium}

\bibitem{sammati} Pyla, Hari K., and Srinidhi Varadarajan: Avoiding deadlock avoidance.
Proceedings of the 19th international conference on Parallel architectures and compilation techniques. ACM, 2010.

\bibitem{rx} F. Qin, J. Tucek, Y. Zhou, and J. Sundaresan. Rx: Treating bugs as
allergiesa safe method to survive software failures. ACM Trans. Comput.
Syst., 25(3), Aug. 2007.

\bibitem{berger} Berger, Emery D., et al: Grace: Safe multithreaded programming for C/C++.
ACM Sigplan Notices. Vol. 44. No. 10. ACM, 2009.

\bibitem{splash} SPLASH-2. SPLASH-2 benchmark suite, available at \url{http://www.capsl.udel.edu/splash}

\bibitem{phoenix} Ranger, Colby, et al: Evaluating mapreduce for multi-core and multiprocessor systems.
High Performance Computer Architecture, 2007. HPCA 2007. IEEE 13th International Symposium on. IEEE, 2007.

\bibitem{grechanik} Grechanik, Mark, et al: Preventing database deadlocks in applications.
Proceedings of the 2013 9th Joint Meeting on Foundations of Software Engineering. ACM, 2013.

\bibitem{sanchez} Iván Sanchez. Latin Squares and Its Applications on Software Engineering. Master's thesis, Federal University of Pernambuco, Recife, Brazil, 2011.

\bibitem{paola} Paola Accioly. Comparing Different Testing Strategies for Software Product Lines. Master's thesis, Federal University of Pernambuco, Recife, Brazil, 2012.

\bibitem{buse} Raymond P. L. Buse, et al. Benefits and barriers of user evaluation in software engineering research. ACM SIGPLAN Notices, 46(10):643-656, October 2011.

\bibitem{runes} Per Runeson. Using students as experiement subjects - an analysis on graduate and freshmen student data. In Proceedings of the 7th International Conference on Empirical Assessment in Software Engineering. Keele University, UK, pages 95-102, 2003.

\bibitem{staron} Miroslaw Staron. Using students as subjects in experiments - A quantitative analysis of the influence of experimentation on students' learning process. In CSEE&T, apges 221-228. IEEE Computer Society, 2007.

\bibitem{box} G. E. P. Box, J. S. Hunter, and W. G. Hunter, Statistics for experimenters: design, innovation, and discovery. Wiley-Interscience, 2005.

\end{thebibliography}