\chapter{Concluding Remarks}
\label{conclusion}

In this work, we have initially discussed how challenging it is to handle deadlocks and resolve them in real world software, and present many previous approaches that
have tried to identify or solve deadlocks with either static analysis, dynamic analysis or a mix of the two approaches.

Given that it's very costly to detect deadlocks in its general form, we've investigated which kind of deadlock is the most popular by looking at actual bug reports
in relevant open source projects and classifying each deadlock manually. We've confirmed a previous study claim that classic deadlocks between two threads and two resources
are the most frequent case of deadlock. This claim allowed us to focus on this specific case instead and minimally modify Java's \emph{ReentrantLock} to provide a lightweight
version of a lock that detects this kind of deadlock in runtime.

However, differently from previous approaches, we provided a runtime exception which should be raised on both threads involved in a \ac{tttl} deadlock
when our Java's \emph{ReentrantLock} implementation is used, as we believe that developers would find bugs faster if deadlocks were just exceptions.

Our usability evaluation with students shows that the presence of deadlock exception indeed reduce the amount of time spent to identify a bug in a code they were not familiar with.
It also shows that less experienced students could identify the source of the deadlock correctly more easily.

\section{Research Contributions}

We can summarize our main contributions as follows:

\begin{enumerate}
\item Investigated deadlock bug reports in open source projects and confirmed a previous study claim that \ac{tttl} deadlocks are the most popular type of deadlock (92.07\% of all resource deadlocks we've identified).
\item Implemented a lightweight version of {\tt ReentrantLock} that detects \ac{tttl} deadlocks and throw runtime exception on both threads.
\item We measured implementation performance overhead with a very conservative benchmark and estimated our cost to be less than 6\% for worse case on real world applications.
\item Presented a sketch of a proof to demonstrate why \ac{tttl} deadlock detection protocol is correct.
\item Empriical evaluation shows that deadlock exceptions help to reduce time spent identifying deadlock bugs.
\item Found evidence that allows us to hypothetize that deadlock exceptions are more helpful for less experienced developers.
\end{enumerate}

\section{Future Work}

We finish this work with some proposals for future work based on some problems we found during this research or opportunities we identified.

\begin{enumerate}
\item 
Significant amount of deadlock bugs we saw during bug reports study used synchronized blocks rather than explicit locks such as {\tt ReentrantLock}.
In order to reach more Java developers, implementing deadlock detection on synchronized blocks should be investigated.
\item
Investigate if it is possible to add more bug context data into {\tt DeadlockException} while still keeping overhead low.
It should be easier to understand more accurately details about deadlock in the exception message there is clear information about which threads were affected and which locks.
\item
We collected data about number of comments and time spent to solve tasks for bug reports we found, but we did not use in this research.
It should help to investigate how difficult deadlock bugs are in comparison to non-deadlock bugs. Are developers discussing more on them, or are they spending more time to close tasks?
\item
Run empirical experiements with other groups of students grouped by experience in concurrent programming development
to confirm whether deadlock exception benefits we saw are indeed more significant for less experienced developers.
\item
Search for deadlock bugs in projects written with other popular languages and investigate if there are significant differences for \ac{tttl} deadlocks distribution.
\end{enumerate}

