\begin{thebibliography}{4}

\bibitem{lu} Lu, Shan, et al: Learning from mistakes: a comprehensive study on real world concurrency bug characteristics.
ACM Sigplan Notices. Vol. 43. No. 3. ACM, 2008.

\bibitem{singhal} Singhal, Mukesh. Deadlock detection in distributed systems.
Computer 22.11 (1989): 37-48.

\bibitem{knapp} Knapp, Edgar: Deadlock detection in distributed databases.
ACM Computing Surveys (CSUR) 19.4 (1987): 303-328.

\bibitem{marino} Marino, Daniel, et al: Detecting deadlock in programs with data-centric synchronization.
Software Engineering (ICSE), 2013 35th International Conference on. IEEE, 2013.

\bibitem{marlow} S. Marlow. Parallel and Concurrent Programming in Haskell: Techniques for Multicore and Multithreaded Programming. O'Reilly, Aug 2013.

\bibitem{golang} M. Aimonetti. Go Bootcamp: Chapter 8 - Concurrency. Chapter available at: \url{http://www.golangbootcamp.com/book/concurrency}

\bibitem{dawson} Dawson Engler and Ken Ashcraft. RacerX: effective, static detection of race conditions and
deadlocks. SIGOPS Operating Systems Review, 37(5):237–252, 2003.

\bibitem{vivek} Vivek K. Shanbhag. Deadlock-detection in java-library using static-analysis. Asia-Pacific
Software Engineering Conference, 0:361–368, 2008.

\bibitem{williams} Amy Williams, William Thies, and Michael D. Ernst. Static deadlock detection for java
libraries. In ECOOP 2005 - Object-Oriented Programming, pages 602–629, 2005.

\bibitem{mcsdk} Da Luo, Zhi, Raja Das, and Yao Qi: Multicore sdk: a practical and efficient deadlock detector for real-world applications.
Software Testing, Verification and Validation (ICST), 2011 IEEE Fourth International Conference on. IEEE, 2011.

\bibitem{magicfuzzer} Cai, Yan, and W. K. Chan: MagicFuzzer: scalable deadlock detection for large-scale applications.
Proceedings of the 2012 International Conference on Software Engineering. IEEE Press, 2012.

\bibitem{sammati} Pyla, Hari K., and Srinidhi Varadarajan: Avoiding deadlock avoidance.
Proceedings of the 19th international conference on Parallel architectures and compilation techniques. ACM, 2010.

\bibitem{serenity} Hari K. Pyla and Srinidhi Varadarajan. Transparent Runtime Deadlock Elimination. In
Proceedings of the 21st international conference on Parallel architectures and compilation
techniques, PACT ’12, pages 477–478, New York, NY, USA, 2012. ACM.

\bibitem{pyla} Pyla, Hari Krishna. Safe Concurrent Programming and Execution. (2013).

\bibitem{valor} Biswas, Swarnendu, et al. Efficient, Software-Only Data Race Exceptions. 2015.

\bibitem{rx} F. Qin, J. Tucek, Y. Zhou, and J. Sundaresan. Rx: Treating bugs as
allergies---a safe method to survive software failures. ACM Trans. Comput.
Syst., 25(3), Aug. 2007.

\bibitem{havelund} Havelund, Klaus, and Thomas Pressburger. Model checking java programs using java pathfinder. International Journal on Software Tools for Technology Transfer 2.4 (2000): 366-381.

\bibitem{repo} Java's ReentrantLock with DeadlockException. Source code: \url{https://github.com/rafaelbrandao/java-lock-deadlock-exception}

\bibitem{orderedlock} Eclipe's OrderedLock class description: \url{http://www.cct.lsu.edu/~rguidry/ecl31docs/api/org/eclipse/core/internal/jobs/OrderedLock.html}

\bibitem{sanchez} Iván Sanchez. Latin Squares and Its Applications on Software Engineering. Master's thesis, Federal University of Pernambuco, Recife, Brazil, 2011.

\bibitem{paola} Paola Accioly. Comparing Different Testing Strategies for Software Product Lines. Master's thesis, Federal University of Pernambuco, Recife, Brazil, 2012.

\bibitem{runes} Per Runeson. Using students as experiement subjects - an analysis on graduate and freshmen student data. In Proceedings of the 7th International Conference on Empirical Assessment in Software Engineering. Keele University, UK, pages 95-102, 2003.

\bibitem{staron} Miroslaw Staron. Using students as subjects in experiments - A quantitative analysis of the influence of experimentation on students' learning process. In CSEE\&T, apges 221-228. IEEE Computer Society, 2007.

\bibitem{box} G. E. P. Box, J. S. Hunter, and W. G. Hunter, Statistics for experimenters: design, innovation, and discovery. Wiley-Interscience, 2005.

\bibitem{agresti} Agresti, Alan. A survey of exact inference for contingency tables. Statistical Science (1992): 131-153.

\bibitem{lozi} Jean-Pierre Lozi, Florian David, Ga\"{e}l Thomas, Julia Lawall, and Gilles Muller. Remote core locking: migrating critical-section execution to improve the performance of multithreaded applications. In Proceedings of the 2012 USENIX Annual Technical Conference (USENIX ATC'12). Berkeley, CA, USA, 2012.

\end{thebibliography}