
\section*{Appendix: Sample Size Calculation In R}

\noindent
\begin{verbatim}
sample.size = function(c.lev, margin=.5,
                       c.interval=.05, population) {
  z.val = qnorm(.5+c.lev/200)
  ss = (z.val^2 * margin * (1-margin))/c.interval^2
  p.ss = round((ss/(1 + ((ss-1)/population))), digits=0)
  METHOD = paste("Recommended sample size for a population of ",
                 population, " at a ", c.lev,
                 "% confidence level", sep = "")
  structure(list(Population = population,
                 "Confidence level" = c.lev,
                 "Margin of error" = c.interval,
                 "Response distribution" = margin,
                 "Recommended sample size" = p.ss,
                 method = METHOD),
            class = "power.htest")
}

sample.size(95, 0.5, 0.05, 541)
\end{verbatim}

\section*{Appendix: Sample Analysis in Python}

\noindent
\begin{verbatim}
import sys

f = open(sys.argv[1], "r")
headers = f.readline().split('\t')
lists = [ list() for i in xrange(4) ]

for line in f.readlines():
  data = line.split('\t')
  data = [ i.strip() for i in data ]
  unit = tuple((data[0], data[2], data[3], data[4], data[5], data[6]))
  group = ord(data[1])-ord('A')
  lists[group].append(unit)

total = 0
overall = len(lists[0]) + len(lists[3])
for u in lists[0]:
  if u[2] == u[3] and u[2] == '2':
    total += 1

print '== results =='
print 'found', str(total), 'deadlock bugs with 2 threads and 2 locks'
print 'found', str(overall), 'potential deadlock bugs'
print 'rate (worse case): ', str(float(total)/float(overall)*100), '%'
best_case = str(float(total+len(lists[3]))/float(overall)*100)
print 'rate (best case): ', best_case, '%'
\end{verbatim}


\section*{Appendix: Fetch Bug Reports Data}

\noindent
\begin{verbatim}
import random
import sys
import datetime

import json
import urllib2
import os.path

f = open(sys.argv[1], "r")
headers = f.readline().split('\t')

path = './bugs/'
bugs = list()

for line in f.readlines():
  data = line.split('\t')
  data = [ i.strip() for i in data ]
  bugs.append(data[0])
  rep,bug_number = data[0].split('-')

  filepath = os.path.join(path, data[0] + '.xml')
  if os.path.exists(filepath):
    continue
  print 'fetching data for', data[0]
  url = ''
  if rep == 'ECLIPSE':
    url = ("https://bugs.eclipse.org/bugs/show_bug.cgi?" +
           "ctype=xml&id=" + bug_number)
  elif rep == 'JDK':
    url = ('https://bugs.openjdk.java.net/si/jira.issueviews:' +
           'issue-xml/' + data[0] + '/' + data[0] + '.xml')
  elif rep == 'LUCENE':
    url = ('https://issues.apache.org/jira/si/jira.issueviews:' +
           'issue-xml/' + data[0] + '/' + data[0] + '.xml')

  u = urllib2.urlopen(url)
  content = u.read()
  out = open(filepath, "wb")
  out.write(content)
  out.close()
  print 'done writing', filepath

def delta_hours(ts1,ts2):
  return round((ts2-ts1).total_seconds() / 60.0 / 60.0,2)

def import_timestamp_jira(str):
  s = str.split()
  s = ' '.join(s[1:5])
  return datetime.datetime.strptime(s, "%d %b %Y %H:%M:%S")

def import_timestamp_bugzilla(str):
  s = str.split()
  s = ' '.join(s[0:2])
  return datetime.datetime.strptime(s, "%Y-%m-%d %H:%M:%S")

def import_from_jira(bug):
  import xml.etree.ElementTree as ET
  tree = ET.parse('./bugs/'+bug+'.xml')
  root = tree.getroot()
  t = root.findall('channel')[0].findall('item')[0]
  created = t.findall('created')[0].text
  resolved = t.findall('resolved')[0].text
  ts1 = import_timestamp_jira(created)
  ts2 = import_timestamp_jira(resolved)
  c = t.findall('comments')
  comments = 0
  if len(c) > 0:
    comments = len(c[0].findall('comment'))
  return tuple((delta_hours(ts1,ts2),comments))

def import_from_bugzilla(bug):
  import xml.etree.ElementTree as ET
  tree = ET.parse('./bugs/'+bug+'.xml')
  root = tree.getroot()
  t = root.findall('bug')[0]
  created = t.findall('creation_ts')[0].text
  resolved = t.findall('delta_ts')[0].text
  ts1 = import_timestamp_bugzilla(created)
  ts2 = import_timestamp_bugzilla(resolved)
  comments = len(t.findall('long_desc'))
  return tuple((delta_hours(ts1,ts2), comments))

for bug in bugs:
  rep = bug.split('-')[0]
  r = tuple()
  if rep == 'ECLIPSE':
    r = import_from_bugzilla(bug)
  else:
    r = import_from_jira(bug)
  print str(r[0]) + '\t' + str(r[1])
\end{verbatim}

\section*{Appendix: Java ReentrantLock pseudocode}

\noindent
\begin{verbatim}
int state;
Thread owner;
Node head;
Node tail;

void lock() {
  if (!tryFastAcquire()) {
    slowAcquire();
  }
}

boolean tryFastAcquire() {
  if (!hasQueuedPredecessors() && COMPARE_AND_SET(state, 0, 1)) {
    setExclusiveOwner(currentThread());
    return true;
  }
  return false;	
}

// Returns true if current thread
// is the first in the queue or it's empty
boolean hasQueuedPredecessors();

void setExclusiveOwner(Thread thread) {
  owner = thread;
}

void slowAcquire() {
  // Creates and atomically enqueue node with current thread
  Node waiterNode = new Node();
  enq(waiterNode);

  // Try a few times to acquire the waiterNode and then park
  // until its predecessor wakes up this thread
  boolean failed = true;
  try {
    while (true) {
      if (waiterNode.pred == head && tryFastAcquire())) {
        setHead(waiterNode);
        failed = false;
        return;
      }
      if (shouldParkAfterFailedAcquire(waiterNode.pred, waiterNode))
        park();
    }
  } finally {
    if (failed)
      cancelAcquire(waiterNode);
  }
}

void release() {
  if (tryRelease()) {
    unparkSuccessor(head);
  }
}

boolean tryRelease(int releases) {
  if (currentThread() != owner)
    return false;
  setExclusiveOwner(null);
  setState(0);
  return true;
}

void park() {
  LockSupport.park(this);
}

// Wakes up successor of a given node in the waiting queue
// if necessary by using LockSupport.unpark on its successor.
void unparkSuccessor(Node);

// Cancel the waiting node and remove from waiting queue.
// If there's a successor parked, unpark it.
void cancelAcquire(Node);

// Atomically checks if the node is really the head
// of the queue and try fastPath codepath.
// On success, the node is dequeued from queue
bool tryFastAcquireIfHead(Node);

// Atomically enqueues node in the waiting queue. It repeately
// tries to COMPARE_AND_SET to update tail until succeeds.
// If head and tail are not initialized yet, there will be
// an extra COMPARE_AND_SET on head to a new Node and tail
// will be set as head.
void enq(Node);

// Make sure to park only when is guaranteed an unpark signal
// can be received. It decides based on specific protocol
// between predecessor of a given node and that node.
shouldParkAfterFailedAcquire(Node, Node);

// Returns Thread corresponding to the current thread
Thread currentThread();

// Disables the current thread for thread scheduling
// purposes unless the permit is available.
LockSupport.park();

// Makes available the permit for the given thread,
// if it was not already available.  If the thread
// was blocked on park then it will unblock.
LockSupport.unpark(Thread);
\end{verbatim}


\section*{Appendix: Instructions in R to evaluate time}
\noindent
\begin{verbatim}
exp1.dat = read.table(file="/Users/rafaelbrandao/r_input.dat", header = T)
attach(exp1.dat)

replica = factor(replica.)
student = factor(student.)
program = factor(program.)
lock = factor(lock.)

# Plot the box plot graphic using the response variable (time)
# associated with the locks with the following command

plot(time~lock,col="gray",xlab="Lock",ylab="Time(seconds)")

# We set the effect model that will serve as basis for posterior analysis.
# Notice that the factor student is associated with the factor replica since for
# each replica we used a different pair of students. We also included the factor
# lock associated with the replica.

anova.ql<-aov(time~replica+student:replica+program+lock+lock:replica)

library(MASS)
bc <- boxcox(anova.ql,lambda = seq(-3, 5, 1/10))
# If transformation is needed, we calculate lambda and use it:
# anova.ql<-aov(time**<lambda>~replica+student:replica+program+lock+lock:replica)
lambda <- bc$x[which.max(bc$y)]

TukeyNADD.QL.REP<-function(objeto1)
{
y1<-NULL
y2<-NULL
y1<- fitted(objeto1)
y2<- y1^2
objeto2<- aov(y2 ~ objeto1[13]$model[,2] +
objeto1[13]$model[,3]:objeto1[13]$model[,2]
+ objeto1[13]$model[,4]+ objeto1[13]$model[,5])
ynew <- resid(objeto1)
xnew <- resid(objeto2)
objeto3 <- lm(ynew ~ xnew)
M <- anova(objeto3)
MSN <- M[1,3]
MSErr <- M[2,2]/(objeto1[8]$df.residual-1)

F0 <- MSN/MSErr
p.val <- 1 - pf(F0, 1,objeto1[8]$df.residual-1)
p.val
}
TukeyNADD.QL.REP(anova.ql)

plot(anova.ql)
anova(anova.ql)
\end{verbatim}

\section*{Appendix: Undergraduate students's time results. Input used in R for analysis}
\noindent
\begin{verbatim}
replica, student, program, lock, time
1, 1, p1, A, 4996
1, 1, p2, B, 5367
1, 2, p1, B, 5070
1, 2, p2, A, 5260
2, 3, p1, A, 2700
2, 3, p2, B, 5306
2, 4, p1, B, 4490
2, 4, p2, A, 4017
3, 5, p1, A, 2340
3, 5, p2, B, 5290
3, 6, p1, B, 3377
3, 6, p2, A, 4473
4, 7, p1, A, 5400
4, 7, p2, B, 5360
4, 8, p1, B, 5400
4, 8, p2, A, 3641
5, 9, p1, A, 5400
5, 9, p2, B, 5400
5, 10, p1, B, 3600
5, 10, p2, A, 2406
6, 11, p1, A, 3290
6, 11, p2, B, 5370
6, 12, p1, B, 5400
6, 12, p2, A, 5320
7, 13, p1, A, 3424
7, 13, p2, B, 5356
7, 14, p1, B, 5400
7, 14, p2, A, 5160
8, 15, p1, A, 2593
8, 15, p2, B, 5279
8, 16, p1, B, 4705
8, 16, p2, A, 4535
9, 17, p1, A, 5160
9, 17, p2, B, 5430
9, 18, p1, B, 5250
9, 18, p2, A, 4246
10, 19, p1, A, 4967
10, 19, p2, B, 5413
10, 20, p1, B, 5400
10, 20, p2, A, 3804
11, 21, p1, A, 5280
11, 21, p2, B, 5160
11, 22, p1, B, 4174
11, 22, p2, A, 4886
12, 23, p1, A, 4271
12, 23, p2, B, 5569
12, 24, p1, B, 5400
12, 24, p2, A, 4788
13, 25, p1, A, 5400
13, 25, p2, B, 5239
13, 26, p1, B, 5310
13, 26, p2, A, 5390
14, 27, p1, A, 2027
14, 27, p2, B, 4271
14, 28, p1, B, 5090
14, 28, p2, A, 4450
15, 29, p1, A, 3000
15, 29, p2, B, 5315
15, 30, p1, B, 5400
15, 30, p2, A, 4210
\end{verbatim}

\section*{Appendix: Graduate students's time results. Input used in R for analysis}
\noindent
\begin{verbatim}
replica, student, program, lock, time
1, 1, p1, A, 1757
1, 1, p2, B, 2404
1, 2, p1, B, 1777
1, 2, p2, A, 1716
2, 3, p1, A, 1342
2, 3, p2, B, 2552
2, 4, p1, B, 2597
2, 4, p2, A, 1238
3, 5, p1, A, 1572
3, 5, p2, B, 2248
3, 6, p1, B, 3168
3, 6, p2, A, 2460
4, 7, p1, A, 1822
4, 7, p2, B, 2455
4, 8, p1, B, 2486
4, 8, p2, A, 2434
5, 9, p1, A, 3503
5, 9, p2, B, 3600
5, 10, p1, B, 2454
5, 10, p2, A, 1753
6, 11, p1, A, 1830
6, 11, p2, B, 3300
6, 12, p1, B, 2880
6, 12, p2, A, 890
7, 13, p1, A, 648
7, 13, p2, B, 940
7, 14, p1, B, 2247
7, 14, p2, A, 1363
\end{verbatim}


