\section{Bug Reports Study}\label{bugs}

Attempting to generalize deadlock detection at runtime does not seem feasible from a performance viewpoint, since existing dynamic analyses take considerable time~\cite{magicfuzzer}. But previous bug reports study~\cite{lu} found that 30 out of 31 deadlock bug reports involved at most two resources. We suspected that TTTL deadlocks were indeed much more common in real world systems than more complex deadlocks, so we investigated this further. This section presents the results of this investigation.

\subsection{Data Collection}

We selected three open source projects to investigate: Lucene, Eclipse and OpenJDK. Lucene\footnote{Lucene: http://lucene.apache.org/} is a text search engine library. Eclipse\footnote{Eclipse: https://eclipse.org/} is one of the most popular IDEs for java developers. OpenJDK\footnote{OpenJDK: http://openjdk.java.net/} is an open-source implementation of the Java Platform. These three projects share some key similarities: they're mostly written in Java; they have immense bug report repositories with easy tools to search into them; and lastly, their bug reports were usually well discussed and contained enough context that allowed us to classify them with some confidence, which was very important in this study.

In total, we collected 541 bug reports containing the word \emph{deadlock} on their titles or on their descriptions. In Lucene, we found 27 closed issues of type "bug" in module "lucene-core".\footnote{Lucene bug reports list: http://goo.gl/DhVI3t}
In Eclipse, we found 406 resolved issues with resolution "fixed".\footnote{Eclipse bug reports: http://goo.gl/qQnrEm}
In OpenJDK, we found 108 issues of type "bug" on module "JDK" with resolution "fixed" and status "resolved".\footnote{OpenJDK bug reports: http://goo.gl/xYFfsO} We then proceeded to calculate the sample size that would allow us to have 95\% of confidence level and 5\% sampling error,  which resulted in 225 bugs. Thus we created a random sample of that size to analyze further \footnote{Bug reports sample: http://goo.gl/zNsIGz}.

\subsection{Data Labeling}

We defined a set of fields to classify for each bug analyzed in the sample. First, we define a category. Then complete other fields based on how much we could understand of each bug report, like the number of threads involved, number of resources involved, type of locking mechanism used, and so on.

We have four different values for the category field.
Category \emph{A} indicates that  we are confident this is a resource deadlock. Thus, we should be able to describe the number of threads and locks that were involved.
In contrast, category \emph{B} represents the opposite: it is certainly not a resource deadlock. The reported bug is a communication deadlock, due to evidence of lost notify/signal in the bug context or anything else that supports it was not a resource deadlock.
Category \emph{C} refers to all the false-positive results: the term \emph{deadlock} was used as a synonym of "hanging" or an "infinite loop", or to just mention another deadlock bug as a reference, not as a cause of the current bug.
Lastly, category \emph{D} is set for all bugs that we could not understand clearly, due to a lack of evidence or discussion.

\subsection{Results Analysis}

Initially we consider only bugs we clearly identified, that is, bugs that were not labeled as category D.
In Table~\ref{tab:categ} (second column), we can see in that from all resource deadlocks, 92.07\% of them are indeed TTTL deadlocks.
Another interesting finding is that 75.93\% of all deadlocks are indeed resource deadlocks.

\begin{table}
\begin{center}
\caption{Labeled Categories and Estimations}\label{tab:categ}
\begin{tabular}{|l|l|l|}
\hline
Category & Number of Bugs & Estimated \\
\hline
A & 101 & 146 \\
A and TTTL & 93 & 134  \\
B & 32 & 46 \\
C & 23 & 33 \\
D & 69 & 0 \\
\hline
\end{tabular}
\end{center}
\end{table}

If we now consider bugs we could not clearly classify, we can make some estimations of how many of them would be resource deadlocks and TTTL deadlocks. The first estimate is the worse case scenario, that is, all bugs in category D should be in category A but none of them would be TTTL deadlocks. In this case, only 54.7\% of resource deadlocks would be TTTL deadlocks. If we look at the best case scenario, that is, all bugs in D would be TTTL deadlocks, then it would be 95.29\% instead. However none of these two scenarios seems realistic. We believe that a more realistic scenario would be to assume that bugs in category D are distributed roughly in the same way as those in categories A, B, and C. If that is the case (last column of Table~\ref{tab:categ}), out of all resource deadlocks, we estimate that 91.7\% of them would also be TTTL deadlocks. Thus TTTL deadlocks are certainly the most popular type of resource deadlocks, amounting to more than 9 out of every 10 resource deadlocks. This result makes it evident that an approach to automatically detect these deadlocks has practical value.  

\subsection{Threats to Validity}

Only one of the authors labeled all bug reports due to constraints on time and lack of resources. In counterpart, having only one reviewer makes it easier to guarantee that all bug reports were reviewed following the exact same procedure, but we would have preferred to have at least one more reviewer to label each bug independently and use it as a way to double check the labels accuracy. Furthermore, one factor that might limit generalization of these findings is that all projects we looked were written in Java and different programming languages may have different distribution of deadlock bugs.
